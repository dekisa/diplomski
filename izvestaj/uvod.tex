\section{Uvod}

Diplomski

socfpga
linux
vhdl
de1soc
cyclone5
alati

Komunikacija linux operativnog sistema prema hardveru realizovanom u FPGA

U ovom radu je na DE1-SoC razvojnom sistemu implementiran jednostavan hardver u FPGA, portovan je Linux operativni sistem i napisan je drajver za pristup registrima i prihvatanje prekida iz FPGA. (Ovo u zakljucak: Izrada ovog rada je bila motivisana zeljom da se upoznaju konkretni alati i postupak projektovanja sa SoCFPGA sistemima. Izvestaj o radu je napisan tako da pruzi kratak pregled bitnih pojmova i detaljna uputsva za reporodukciju rezultata sa osvrtom na usputne probleme. Steceno znanje moze olaksati slusanje predmeta na master studijama, ili biti osnova za resavanje konkretnog problema (npr. ubrzavanje algoritama kompresije i obrade slike), a autoru je olaksalo snalazenje na novom radnom mestu. Autor izrazava veliku zahvalnost profesoru Lazaru Saranovcu i asistentu Strahinji Jankovicu za podrsku i savete prilikom izrade diplomksog rada.

Sa porastom mogućnosti namenskih sistema došlo je do popularizacije sistema na čipovima (SoC - System on Chip) koji integrišu mikroprocesore sa više jezgara, memorije na čipu, mnogobrojne periferije i transivere, kao i FPGA (Field Programmable Gate Array).

Ova tehnologija daje dizajneru sistema veliku slobodu i mogućnosti. Mogućnosti klasičnog pristpa projektovanju namenskih sistema su proširene punom snagom FPGA. Uz to se ostvaruje veća integracija, manja potrošnja, manja površina štampane ploče(PCB - Printed Circuit Board) i veći protok podataka između procesora i FPGA dela. 

Uobičajena primena ovih sistema je implementacija specifičnih akceleratora koji ubrzavaju izvršavanje algoritama i implementacija specifičnih programabilnih interfejsa ka spoljnom svetu. Nove tehnologije kao što su OpenCL, (navedi jos)omogućavaju kompatibilnost dizajna softvera na visokom nivou i implementiranog hardvera na niskom nivou.

SoCFPGA sistemi najčešće sadrže ARM mikroprocesor. Aplikacije na mikroprocesoru bez operativnog sistema (baremetal) nude jednostavno pisanje koda i uštedu na resursima. Za kompleksnije aplikacije koriste se operativni sistemi (OS) i time se olakšava integrisanje mrežnih protokola, rad sa multimedijalnim sadržajima, kriptografskim bibliotekama kao i mnoge druge mogućnosti koje su dostupne kao open-source softver. Kada je potrebno garantovati reakciju u određenom vremenu na neki spoljni događaj veliki operativni sistemi nisu dobro rešenje i koriste operativni sistemi u realnom vremenu (RTOS).

Hardver u FPGA se projektuje upotrebom nekog dva popularna jezika za opis hardvera (Verilog i VHDL - Very High Speed Integrated Circuit Hardwer Description Language) i softverskih alata za specifični uređaj. Dodatno ovi alati olakšavaju dizajn upotrebom IP( Intelectual Property) blokova kao i generisanjem raznih izlaznih fajlova koji opisuju projektovani hardver na standardni način i koriste se prilikom razvoja softvera.

U ovom radu korišćen je DE1-SoC razvojni sistem koji se vrlo često upotrebljava u edukativne svrhe. Razvojni sistem je zasnovan na čipu iz familije Cyclone 5 kompanije Altera (sada Intel) sa dodatom DDR3 memorijom kapaciteta 1GB i hardverom za audio, video, mrežu i ostalo. Korišćeni alati za razvoj su Quartus, Altera EDK(Embedded Developmnent Kit), i mnogi drugi koji će biti pomenuti u uputsvu.

Kratak opis Cyclone 5 sa procesom boot-ovanja.
Detaljan opis projekta i svih delova
