\section{Zaključak}
Ovaj projekat je doneo uvid kako u kompleksnost korišćenih uređaja tako i u široke mogućnosti njihove primene. Pri radu na projektu su uočene mogućnosti za proširivanje  projekta (mogu se nazvati i manama projekta).\\

Jedan od nedostatka projekta je obrada komunikacije sa iWrap softverom. Naime iWrap podržava mnogobrojne komande koje se tiču konfigurisanja module (npr. konfiguracija bezbednosti, inicijalizacija konekcije) i šalje mnogobrojna obaveštenja (npr. o dostupnim uređajima, izgubljenoj konekciji). Idealno bi bilo obezbediti interfejs za slanje komandi i mogućnost reagovanja n obaveštenja. Iako je implementacija ovog interfejsa jednostavna, pokazalo se da je u praktičnom radu komplikovano doneti odluku o tome šta je potrebno da se uradi da bi se ostvarila konekcija (upisivanjem direktno na AT komandnu liniju se pokazalo da konfiguracija i ostvarivanje zahteva svega tri komande ali i veliki broj pokušaja, najverovatnije usled nešto komplikovanije obrade uspostavljanja veze sa strane računara). Stoga je u ovom projektu jednom izvršeno uparivanje sa test računarom (koje je sačuvano na modulu i na računaru), isključena su sva obaveštenja koja šalje modul. Tako se konekcija ostvaruje jednostavnim otvaranjem COM porta, odakle se terminal s računarske strane i UART sa mikrokontrolerske strane ponašaju kao najobičnija serijska komunikacija.

Takođe, za debaunsiranje dugmića i za osvežavanje displeja je bilo prikladnije koristiti neki od dostupnih tajmera umesto pozivanja iz glavne petlje.